% XeLaTeX can use any Mac OS X font. See the setromanfont command below.
% Input to XeLaTeX is full Unicode, so Unicode characters can be typed directly into the source.

% The next lines tell TeXShop to typeset with xelatex, and to open and save the source with Unicode encoding.

%!TEX TS-program = xelatex
%!TEX encoding = UTF-8 Unicode

\documentclass[12pt]{article}
\usepackage{geometry}                % See geometry.pdf to learn the layout options. There are lots.
\geometry{letterpaper}                   % ... or a4paper or a5paper or ... 
%\geometry{landscape}                % Activate for for rotated page geometry
%\usepackage[parfill]{parskip}    % Activate to begin paragraphs with an empty line rather than an indent
\usepackage{graphicx}
\usepackage{amssymb}
\usepackage{amsmath}
% Will Robertson's fontspec.sty can be used to simplify font choices.
% To experiment, open /Applications/Font Book to examine the fonts provided on Mac OS X,
% and change "Hoefler Text" to any of these choices.

\usepackage{fontspec,xltxtra,xunicode}
\defaultfontfeatures{Mapping=tex-text}
%\setromanfont[Mapping=tex-text]{Hoefler Text}
\setsansfont[Scale=MatchLowercase,Mapping=tex-text]{Gill Sans}
\setmonofont[Scale=MatchLowercase]{Andale Mono}

\title{Introduction To Algorithms \\ HomeWork 1 Solutions}
\author{Junzhe Zheng}
%\date{}                                           % Activate to display a given date or no date

\begin{document}
\maketitle

\paragraph{1.}
\hangafter 1
\hangindent 1.85em
\noindent
I used Matlab to solve this problem and got result is 30.
\\Code:
\\\indent n=1;
\\\indent while 200*n\^{}2 > 1.5\^{}n
\\\indent\indent    n = n + 1;
\\\indent end
\\\indent disp(n);

\paragraph{2.}
\hangafter 1
\hangindent 1.85em
\noindent
(a) is False.
\\For example, $f(n)=log(n)$ and $g(n)=n^3$. By assuming that $g(n)=\hat{O}(f(n))$, we have:
\[
	0\leq n^3\leq c(log(n))^2,  \quad  \forall n>n_0
\]
For all $n>1$, this does not hold. Thus the assumption is wrong,
\\
\\
\\
\\(b) is conditional True. 
\\$f(n)=\hat{O}(g(n))$, by definition, we have:
\[
	0\leq f(n)\leq cg(n)log(n), \quad \forall n>n_0
\]
For $f(n)\geq1$,taking the logarithm on both side, we get:
\[
	0\leq log(f(n))\leq logc+log(g(n))+log(log(n)), \quad \forall n>n_0
\]
If $log(g(n))\geq1$ for sufficiently large n, we will have:
\[
	0\leq log(f(n))\leq logc*log(g(n))+log(g(n))+log(g(n))*log(log(n)), \quad \forall n>n_0
\]
Because n is very large, then $log(n)\geq log(log(n))$ and $log(n)\geq 1$, we get
\[
	0\leq log(f(n))\leq logc*log(g(n))*log(n)+log(g(n))*log(n)+log(g(n))*log(n), \quad \forall n>n_0
\]
then:
\[
	0\leq log(f(n))\leq (logc+2)*log(g(n))*log(n), \quad \forall n>n_0
\]
\[
	0\leq log(f(n))\leq c_2*log(g(n))*log(n), \quad \forall n>n_0
\]
By definition, this implies $log(f(n))=\hat{O}(log(g(n)))$.
\\Thus, this is True for $f(n)\geq1$,$log(n)\geq1$ and n is a very large number.
\\
\\
\\
(c) is True.
\\By definition, $o(f(n))$ implies:
\[
	0\leq g(n) < cf(n), \quad \forall n > n_0 \quad and \quad \forall c > 0
\]
Both sides plus $f(n)>0$, we get:
\[
	0\leq f(n) \leq f(n)+g(n) < (c+1)f(n), \quad \forall n > n_0 \quad and \quad \forall c > 0
\]
\[
	0\leq f(n) \leq f(n)+g(n) < c_2f(n), \quad \forall n > n_0 \quad and \quad \forall c_2 > 1
\]
But we can choose a smaller $c_2$ to make:
\[
	0\leq f(n) \leq f(n)+g(n) \leq c_2f(n), \quad \forall n > n_0 \quad and \quad \forall c_2 > 0
\]
By definition, this implies $f(n)+g(n)=\Theta(f(n))$.
\\
\\
\\
(d) is False.
\\Let $f(n)=1$ and $g(n)=n$.Then $f(n)+g(n)=1+n\leq O(min(f(n),g(n)))=O(f(n))=cf(n)=c$,for $\forall n\geq n_0$,which is impossible.

\paragraph{3.}
\hangafter 1
\hangindent 1.85em
\noindent
For a N floor building, drop first egg at floor x. This can lead to two results:1.Break. Then we should start at ground floor to x floor, worst case is x. 2 Not Break. Then we should go up x-1 floor from x, which leads us to two situations. If break, then start at x+1 floor to x+(x-1) floor, worst case x-1. But in our previous test, we had dropped egg at x floor , so the worst case for this is still x-1+1=x. If Not break, we just recursively do above procedure with 1 less floor when go up till x =1.  Here we have:
\[
n+(n-1)+(n-2)+ \cdots +3+2+1\geq N
\]
we solve for:
\[
\frac{(n+1)*n}{2}= N
\]
Because floor number is a positive integer, solution is:
\[
n=\lceil\frac{-1+\sqrt{1+8N}}{2}\rceil 
\]
\end{document}  

