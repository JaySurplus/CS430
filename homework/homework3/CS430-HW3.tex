% XeLaTeX can use any Mac OS X font. See the setromanfont command below.
% Input to XeLaTeX is full Unicode, so Unicode characters can be typed directly into the source.

% The next lines tell TeXShop to typeset with xelatex, and to open and save the source with Unicode encoding.

%!TEX TS-program = xelatex
%!TEX encoding = UTF-8 Unicode

\documentclass[11pt]{article}
\usepackage{geometry}                % See geometry.pdf to learn the layout options. There are lots.
\geometry{letterpaper}                   % ... or a4paper or a5paper or ... 
%\geometry{landscape}                % Activate for for rotated page geometry
%\usepackage[parfill]{parskip}    % Activate to begin paragraphs with an empty line rather than an indent
\usepackage{graphicx}
\usepackage{amssymb}
\usepackage{amsmath}
% Will Robertson's fontspec.sty can be used to simplify font choices.
% To experiment, open /Applications/Font Book to examine the fonts provided on Mac OS X,
% and change "Hoefler Text" to any of these choices.

\usepackage{fontspec,xltxtra,xunicode}
\defaultfontfeatures{Mapping=tex-text}
%\setromanfont[Mapping=tex-text]{Hoefler Text}
\setsansfont[Scale=MatchLowercase,Mapping=tex-text]{Gill Sans}
\setmonofont[Scale=MatchLowercase]{Andale Mono}

\title{Introduction To Algorithms \\ HomeWork 3 Solutions}
\author{Junzhe Zheng}
%\date{}                                           % Activate to display a given date or no date

\begin{document}
\maketitle

\paragraph{1.}
\hangafter 1
\hangindent -2em
\noindent
\\
\begin{tabular}{|c|c|c|c|c|c|}
\hline
Size  &RT(ms) & RT Ratio & nlog(n) &nlog(n) Ratio &Error\\
\hline
$2.5*10^{7}$ & 291 &   & $2.5*10^{7}*21.25$ &  &\\
\hline
$5*10^{7}$    & 591 & $\frac{591}{291}=2.03$ & $5*10^{7}*22.25$ & $\frac{5*10^{7}*22.25}{2.5*10^{7}*21.25}=2.09$ &$\frac{2.03-2.09}{2.09}=-2.87\%$ \\
\hline
$1*10^{8}$   & 1235 & $\frac{1235}{591}=2.09$ & $1*10^{8}*23.25$   & $\frac{1*10^{8}*23.25}{5*10^{7}*22.25}=2.09$ & $\frac{2.09-2.09}{2.09}=0.00\%$  \\
\hline
$2*10^{8}$   & 2567 & $\frac{2576}{1235}=2.09$ & $2*10^{8}*24.25$ & $\frac{2*10^{8}*24.25}{1*10^{8}*23.25}=2.09$ & $\frac{2.09-2.09}{2.09}=0.00\%$ \\
\hline
$4*10^{8}$   & 5416 & $\frac{5416}{2567}=2.11$ & $4*10^{8}*25.25$ & $\frac{4*10^{8}*25.25}{2*10^{8}*24.25}=2.08$ & $\frac{2.11-2.08}{2.08}=1.44\%$\\
\hline
$8*10^{8}$   & 11073 & $\frac{11073}{5416}=2.04$ & $8*10^{8}*26.25$ & $\frac{8*10^{8}*26.25}{4*10^{8}*25.25}=2.08$ & $\frac{2.04-2.08}{2.08}=-1.92\%$\\
\hline
\end{tabular}
\\
\\Let {n} be a list of integer number,$n_1<n_2<n_3<n_4<n_5<n_6$.
\\Define:
\[
	RT\  Ratio\ =\ \frac{n_{i+1}}{n_{i}}	
\]
\[
	nlog(n)\ Ratio\ =\ \frac{n_{i+1}*log(n_{i+1})}{n_i*log(n_i)}
\]
\[
	Error\ =\ \frac{RT\ Ratio\ -\ nlog(n)\ Ratio}{nlog(n)\ Ratio}
\]
\\From above table, we can notice that the Error are small, which indicate that time complexity of Quicksort is based on n*log(n).
\end{document}  

