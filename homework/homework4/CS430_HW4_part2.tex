% XeLaTeX can use any Mac OS X font. See the setromanfont command below.
% Input to XeLaTeX is full Unicode, so Unicode characters can be typed directly into the source.

% The next lines tell TeXShop to typeset with xelatex, and to open and save the source with Unicode encoding.

%!TEX TS-program = xelatex
%!TEX encoding = UTF-8 Unicode

\documentclass[11pt]{article}
\usepackage{geometry}                % See geometry.pdf to learn the layout options. There are lots.
\geometry{letterpaper}                   % ... or a4paper or a5paper or ... 
%\geometry{landscape}                % Activate for for rotated page geometry
%\usepackage[parfill]{parskip}    % Activate to begin paragraphs with an empty line rather than an indent
\usepackage{graphicx}
\usepackage{amssymb}
\usepackage{amsmath}
% Will Robertson's fontspec.sty can be used to simplify font choices.
% To experiment, open /Applications/Font Book to examine the fonts provided on Mac OS X,
% and change "Hoefler Text" to any of these choices.

\usepackage{fontspec,xltxtra,xunicode}
\defaultfontfeatures{Mapping=tex-text}
%\setromanfont[Mapping=tex-text]{Hoefler Text}
\setsansfont[Scale=MatchLowercase,Mapping=tex-text]{Gill Sans}
\setmonofont[Scale=MatchLowercase]{Andale Mono}

\title{Introduction To Algorithms \\ HomeWork 4 Solutions}
\author{Junzhe Zheng}
%\date{}                                           % Activate to display a given date or no date

\begin{document}
\maketitle

\paragraph{1.Give an example where quicksort requires $O(n^{2})$ steps.}
\hangafter 1
\hangindent 1.1em
\noindent
\\Consider a list:
\[
	10,9,8,7,6,5,4,3,2,1
\]
We choose the last digit in the last as the pivot.Thus time complexity is given as:
\[
	T(n)\ =\ T(n-1)+T(0)+\Theta(n)
\]
By using substitution method, we could get:
\[
	T(n)\ =\ \Theta(n^{2})
\]
If it is $\Theta(n^{2})$, it is also a $O(n^{2})$.
\\
\paragraph{2.Problem 4-6 (Page 110) CLRS(3rd Edition).}
\hangafter 1
\hangindent 1.1em
\noindent
\\a. Need to prove "if and only if", thus the proof will have to separate parts
\\\it{ Proof of `Only if'}\normalfont:
\\If A is a Monge array, by definition, we have:
$$A[i,j]\ +A[k,l]\ <\ A[i,l]\ +A[k,j]\ \ \forall i\ ,\ j\ ,\ k\ ,\ l$$
$$where\ 1<i<k<n\ ,\ 1<j<l<m$$
Let $k=i+1,\ l=j+1$, we will have:
$$A[i,j]\ +A[i+1,j+1]\ <\ A[i,j+1]\ +A[i+1,j]\ \ \forall i\ ,\ j$$
$$where\ 1<i<i+1<n\ ,\ 1<j<j+1<m$$
$$where\ 1<i<n-1\ ,\ 1<j<m-1$$
`Only if' has been proved.
\\
\\\it{Proof of `if'}\normalfont:
\\Induction method will be used separately on rows and columns.
\\For rows:
\\
\paragraph{3.Using the version of heap sort as defined in CLRS(chapter 6-4),show an example where heapsort requires $\Omega$(nlogn) steps. }
\hangafter 1
\hangindent 1.1em
\noindent
\\For an array that each elements are already sorted in an increasing order, the performance of heapsort is $\Omega (nlg(n))$. Because that each of the n-1 calls of Max\_HEAPIFY (for i= A.length downto 2)takes $\Omega (lg(n))$.
\\
\paragraph{4.Consider radix sort with numbers (using base 10 ) that are variable length. Show that you can output any number as soon as you have considered all its digits.
Design a method to sort in O(n + k) time where k is the total number of digits in all the numbers.}
\hangafter 1
\hangindent 1.1em
\noindent
\\For radix sort we start at checking the least digits. At the $i^{th}$ digits sorting process, the number are ordered by the $i^{th}$ least digits.
\\For a number with length j, once we finished checking $j^{th}$ digits, we could output it in the right position in an sorted array.
\\To output a number, we need to check every number remain in to-be-sort list after $l^{th}$ iteration. For a number with length i, if i=l, we output this number to sorted list. If i>l, then put this number to l+1 bucket for $l+1^{th}$ iteration.
\\There are n number and k is the total length over all n digits. Thus the time complexity is  $O(n+k)$.
\end{document}  

